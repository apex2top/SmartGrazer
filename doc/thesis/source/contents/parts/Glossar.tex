\newglossaryentry{tracken}{name=tracken, description={Verfolgen/Nachverfolgen von Benutzern oder Daten im Cyberspace.}}

\newglossaryentry{payload}{name=Payload, description={Ein Stück Quellcode, bestehend aus HTML- und JavaScript-Code, welches beim Verarbeiten vom Browser auf der Seite ausgeführt werden soll.}}

\newglossaryentry{XSS}{name=XSS, description={Abkürzung für \textbf{\acl{XSS}}.\newline Die Abkürzung \acs{XSS} wird verwendet um Missverständnisse mit der Auszeichnungssprache \ac{CSS} zu vermeiden.}}

\newglossaryentry{Crawler}{name=Crawler, description={Ein (Web-)Crawler ist eine Softwareanwendung, die systematisch alle Links einer Webseite besucht und herunterläd bzw. auswertet.}}

\newglossaryentry{Backtick}{name=Backtick, description={Rückwärts geneigtes Hochkomma.}}

\newglossaryentry{Sanitizer}{name=Sanitizer, description={Die Daten werden zwischen Eingabe und Verarbeitung untersucht und von verbotenen Inhalten ``gereinigt''.}}

\newglossaryentry{Eventhandler}{name={Eventhandler},description={Eventhandler sind, im Kontext von HTML, Einstiegspunkte um JavaScript-Code auszuführen.}}

\newglossaryentry{Keylogger}{name=Keylogger, description={Ein Keylogger ist eine Softwareanwendung, die sämtliche Anschläge der Tastatur aufzeichnet und (meistens) an eine dritte Person übermittelt.}}

\newglossaryentry{GET}{name=GET, description={Eine Methode des \ac{HTTP} bzw. \ac{HTTPS}, um Daten vom Browser an den Webserver zu senden. GET-Parameter werden beim Übermitteln an die \ac{URL} angehängt und können somit direkt übertragen werden.}}

\newglossaryentry{POST}{name=POST, description={Eine Methode des \ac{HTTP} bzw. \ac{HTTPS}, um Daten vom Browser an den Webserver zu senden. POST-Parameter werden als versteckte Daten übertragen und können nicht einfach an die URL angehängt werden.}}

\newglossaryentry{Cookies}{name=Cookies, description={Eine vom Webbrowser angelegte Datei, in der beispielsweise Anmeldeinformationen gespeichert werden, sodass sich der Benutzer bei einem erneuten Besuch der Webseite nicht erneut anmelden muss.}}

\newglossaryentry{Phishing}{name=Phishing, description={Eine Betrugsmasche, bei der Benutzer auf gefälschte Webseiten gelockt werden, um dort Anmeldeinformationen abzugreifen.}}

\newglossaryentry{JSON}{name=JSON, description={ Abkürzung für \textbf{\acl{JSON}}.\newline Das JSON-Format ist ein einfach zu lesendes und kompaktes Datenformat. Der Vorteil von JSON-Daten ist, dass diese oft direkt als Objekte interpretiert und verwendet werden können.}}

\newglossaryentry{AJAX}{name=AJAX, description={ Abkürzung für \textbf{\acl{AJAX}}.\newline Eine Technik, bei der unter Verwendung von JavaScript neue Serveranfragen auch nach dem vollständigen Laden der Webseite getätigt werden können, ohne die komplette Webseite neu laden zu müssen.}}

\newglossaryentry{Adapter}{name=Adapter, description={Das Adapter-Entwurfsmuster übersetzt die Schnittstelle eines Programms, sodass zwei Anwendungen mit einander kommunizieren können.}}

\newglossaryentry{BNF}{name=BNF, description={Abkürzung für \textbf{\acl{BNF}}.\newline Eine Notations- oder Darstellungsform für kontextfreie Grammatiken. In der \acl{BNF} können auch höhere Programmiersprachen, wie z.B. Java oder Pascal, dargestellt werden.}}
