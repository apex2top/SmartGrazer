\chapter{Einleitung}\label{Einleitung}

\section{Die Nutzung des Internets}

Laut einer Studie von We Are Social Singapore aus dem Jahr 2016 nutzen ca 46\% (3.419 Milliarden Menschen) der Weltbevölkerung das Internet \cite{WeAreSocial2016}. Im Vergleich zum Vorjahr entspricht dies einem Zuwachs von zehn Prozent. In Europa beträgt der Anteil der Internetnutzer 73\% (616 Millionen Menschen) der Gesamtbevölkerung. Weiterhin sind aktuell ca. 189 Millionen \acp{TLD} im Internet registriert \cite{RegistrarStats2017}. Mit der Einführung von Web 2.0 und der vereinfachten Weise, Inhalte im Internet bereit zu stellen, wächst intuitiv auch die Gefahr, dass die Zahl von Webseiten mit Sicherheitslücken zunimmt.

\section{Motivation}
Moderne Webanwendungen sind in der Regel hochgradig dynamisch und interaktiv. Jedoch kann durch bestimmte Benutzereingaben das Verhalten der Webanwendung beeinträchtigt werden. Solche Eingaben können dazu genutzt werden, die Datenbank der Webanwendung zu manipulieren (SQL-Injection) oder schädlichen Quellcode auf den Computern der Besucher der Seite auszuführen (\gls{XSS}).

Laut einer statistischen Analyse von WhiteHat Security \cite{WhiteHatSecurity2016} machen \acl{XSS} Angriffe rund die Hälfte der Angriffe auf Webseiten aus. Weiterhin ist die Anzahl von \acl{XSS}-Angriffen seit 2012 stetig gestiegen. Daher wird es immer wichtiger, diesem Trend entgegen zu wirken und Webseiten ausreichend gegen solche Angriffe zu schützen. Insbesondere müssen sicherheitsrelevante Eingabefelder identifiziert und auf Schwachstellen untersucht werden.

Getestet werden können solche kritischen Eingabefelder oft nur bedingt, da Firmen aus Zeit- oder Budgetgründen auf Software von anderen Quellen zurückgreifen. Dies erschwert das Testen dahingehend, dass der Quellcode oft nicht verfügbar ist und deswegen auf sogenanntes Black-Box-Testing zurückgegriffen werden muss. Auch wenn der Quellcode zugänglich ist, wird oft nur ergänzend mittels Source-Code-Audit getestet, da dies sehr aufwändig ist.

Diese Vorgehensweise entspricht im Normalfall auch den Herausforderungen, denen sich Angreifer annehmen müssen. Aus diesem Grund pflegen Sicherheitsexperten Listen mit Angriffssignaturen, um häufige Fehlerquellen zu identifizieren.

Gängige Softwarelösungen bieten Listen mit geläufigen Angriffssignaturen, die gegen Webanwendungen getestet werden können. Alternativ pflegt die Organisation \ac{OWASP} seit September 2012 eine frei zugängliche Liste mit möglichen Angriffen, um \ac{XSS}-Filter zu umgehen.
Ein Nachteil solcher statischer Listen ist, dass Entwickler von Webanwendungen oder \acp{WAF} Gegenmaßnahmen gegen genau diese Fälle programmieren können.

Des Weiteren können Scanner mit vordefinierten Listen nicht auf gegebene Spezifika der Webanwendung reagieren. Ein weiterer Nachteil der Abarbeitung von umfangreichen statischen Listen ist der Aufwand, der entsteht, wenn alle Angriffssignaturen der Liste komplett abgearbeitet werden sollen.

\section{Aufgabendefinition}

Diese Masterthesis besteht aus insgesamt drei Teilaufgaben:
\begin{itemize}
	\item Im ersten Schritt wird ein Algorithmus entwickelt und implementiert, welcher unter Verwendung von Fuzzing Angriffssignaturen generiert. Hierbei wird ein gezieltes, “intelligentes” Fuzzing angewendet, das auf Konstruktionsvorschriften und Mutationen von Testfällen basiert. Die Implementierung dieses Algorithmus soll parametrisiert steuerbar sein, damit gegebenenfalls bestimmte Muster der Angriffssignatur mit einer höheren Wahrscheinlichkeit generiert werden können.
	\item Weiterhin sollen Antworten von Webanwendungen ver- und bewertet werden, sodass potenziell vielversprechende Angriffssingnaturmuster bei der Weiterentwicklung für den jeweiligen Parameter bevorzugt wiederverwendet werden können.
	\item Im letzten Schritt und nach der Implementierung wird die Anwendung anderen Lösungen gegenübergestellt und Analysen für verschiedene Webapplikationen durchgeführt.
\end{itemize}

\paragraph{Anmerkung:}
Sämtliche Artefakte, wie zum Beispiel Quelltext, Kommentare und Diagramme, werden in englischer Sprache verfasst, da diese nach Abschluss der Masterthesis als Open Source Projekt veröffentlicht werden. Die in dieser Arbeit vorgestellten Quelltext-Passagen und Diagramme werden dementsprechend ausreichend beschrieben.

\subsection{Grenzen}

Der primäre Fokus dieser Masterthesis liegt auf reflektiertem \acl{XSS} (XSS-1). Die Erweiterung auf persistentes \acl{XSS} (XSS-2) kann ohne weitere Anpassungen des hier entwickelten Algorithmus vorgenommen werden. Die Implementierung des Algorithmus wird zunächst als Kommandozeilenprogramm ohne Benutzeroberfläche realisiert.

\subsection{Auswertung}

Getestet wird der entwickelte Algorithmus im direkten Vergleich zu bereits existierenden Schwachstellen-Scannern. Jede Anwendung wird anhand mehrerer Testdurchläufe mit verschiedenen Webanwendungen auf Effektivität und Effizienz bewertet.